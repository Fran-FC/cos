% !TeX spellcheck = es_ES
\documentclass[]{article}

\usepackage{hyperref}
\usepackage[utf8]{inputenc} 


\makeatletter
\setlength{\@fptop}{0pt}
\makeatother

\setlength{\parindent}{0pt}
\setlength{\parskip}{1ex plus 0.5ex minus 0.2ex}


%opening
\title{Kubernetes - Configuración y optimización de sistemas de cómputo }
\author{Francesc Folch Company}

\begin{document}

\maketitle

\begin{abstract}

\end{abstract}

\section{Kubernetes (K8S)}

\subsection{Linux cgroups y namespaces}

Los cgroups (grupos de control)\cite{cgroups} son una funcionalidad del kernel de Linux que permite a los procesos ser organizados y separados mediante una jerarquía grupal. Esto habilita la compartimentalización de recursos, ya que se puede limitar el uso de CPU y memoria a unos procesos de un mismo cgroup.

Los namespaces \cite{namespaces}  
 

\begin{thebibliography}{10}
\bibitem{cgroups}
	cgroups(7). \textit{Linux Programmer's Manual}.
	
	<\url{https://www.man7.org/linux/man-pages/man7/cgroups.7.html}> 
	\newblock [Consulta: 28 de enero de 2022]

\bibitem{namespaces}
namespaces(7). \textit{Linux Programmer's Manual}.

<\url{https://man7.org/linux/man-pages/man7/namespaces.7.html}> 
\newblock [Consulta: 28 de enero de 2022]


\end{thebibliography}
\end{document}
